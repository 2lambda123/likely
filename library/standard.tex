 % % % % % % % % % % % % % % % % % % % % % % % % % % % % % % % % % % % % % % %
 % Copyright 2014 Joshua C. Klontz                                           %
 %                                                                           %
 % Licensed under the Apache License, Version 2.0 (the "License");           %
 % you may not use this file except in compliance with the License.          %
 % You may obtain a copy of the License at                                   %
 %                                                                           %
 %     http://www.apache.org/licenses/LICENSE-2.0                            %
 %                                                                           %
 % Unless required by applicable law or agreed to in writing, software       %
 % distributed under the License is distributed on an "AS IS" BASIS,         %
 % WITHOUT WARRANTIES OR CONDITIONS OF ANY KIND, either express or implied.  %
 % See the License for the specific language governing permissions and       %
 % limitations under the License.                                            %
 % % % % % % % % % % % % % % % % % % % % % % % % % % % % % % % % % % % % % % %

\documentclass[twoside=false, numbers=noenddot]{scrbook}

% The Likely tokenizer will look for the use of the "likely" environment to
% determine what parts of the document are Likely source code.
\newenvironment{likely}
{ \verbatim }
{ \endverbatim }

\usepackage{booktabs}
\usepackage[T1]{fontenc}
\usepackage[hidelinks]{hyperref}
\usepackage{listings}
\usepackage{lmodern}
\usepackage{makeidx}
\usepackage{scrhack}
\usepackage{tocbibind}
\usepackage{verbatim}

\lstset{basicstyle=\ttfamily,
  showstringspaces=false
}

\makeindex

\newcommand{\noun}[1]{\textsc{#1}}
\newcommand{\cref}[3]{\href{#1}{#2}\cite{#3}}
\newcommand{\fref}[2]{\href{#1}{#2}\footnote{\href{#1}{#1}}}
\newcommand{\awsurl}{https://s3.amazonaws.com/liblikely/}
\newcommand{\githuburl}{https://raw.githubusercontent.com/biometrics/likely/gh-pages/}
\newcommand{\dindex}[1]{\textit{#1}\index{#1}}
\newcommand{\pindex}[1]{#1\index{#1}}
\newcommand{\tindex}[1]{\pindex{\texttt{#1}}}

% Acronyms
\newcommand{\AST}{AST\index{abstract syntax tree (AST)}}
\newcommand{\CTFE}{CTFE\index{compile-time function evaluation (CTFE)}}
\newcommand{\GFM}{GFM\index{GitHub Flavored Markdown (GFM)}}
\newcommand{\JIT}{JIT\index{just-in-time (JIT)}}
\newcommand{\LLVM}{LLVM\index{Low Level Virtual Machine (LLVM)}}

\hyphenation{Java-Script}

\title{The Likely Programming Language \\
       {\large Language Reference and Standard Library}}
\author{Joshua C. Klontz}
\date{\today}

\begin{document}
\maketitle
\tableofcontents

\chapter{Introduction}
\fref{http://www.liblikely.org/}{\noun{Likely}} is a programming language for image processing and statistical learning algorithm development.
The following sections discuss the guiding design decisions behind Likely.

\section{Embeddable Compiler}
Likely is foremost a software library.
Both the \fref{\awsurl doxygen/console\_interpreter\_compiler.html}{\texttt{likely} console-interpreter-compiler} and the \fref{\awsurl doxygen/integrated\_development\_environment.html}{\texttt{dream} integrated development environment} are built on top of the \fref{\awsurl doxygen/index.html}{\noun{Likely C API}}.

Understandably, there is more involved in deploying a computer vision application than applying algorithms to pixels alone.
Likely is designed to excel at pixel/feature/subspace transformations, while leaving general application logic for the developer's programming language of choice.
Likely code can both call, and be called by the \cref{http://www.cs.otago.ac.nz/cosc440/readings/C-Programming-Ebook.pdf}{C Programming Language}{kernighan1988c}, and by extension many other programming languages as well.

It was with this focus on an embeddable compiler that the \url{liblikely.org} domain name was chosen.

\section{Compile-Time Function Evaluation}
Compile-time function evaluation (\CTFE) is a core tenant of the Likely programming language.
Likely rejects the learning of a statistical model as an offline step after algorithm compilation.
Instead, \emph{Likely formulates statistical learning as the compile-time evaluation of a function} (the learning algorithm) \emph{with constant arguments} (the training data).

Likely leverages the \cref{http://llvm.org/}{Low Level Virtual Machine}{lattner2004llvm} (\LLVM) project for both static and just-in-time (\JIT) compilation.
One particularly salient use of the \JIT\ compiler is for \CTFE, as the \JIT\ compiler automatically leverages all instruction set extensions and coprocessors available at runtime.
In the special case where the function to be executed does not have loops, Likely will instead run an interpreter, which has been found to be faster for executing simple code.

A computer vision algorithm is a series of image processing and statistical learning transformations requiring incoming samples that conform to particular type and dimensionality expectations.
Likely exploits these expectations through inference on the output of functions evaluated at compile time.
Algorithms written without specifying parameter types are still statically compiled because \emph{\CTFE\ output dictates parameter type, and often array dimensionality as well}.
This is in contrast to conventional computer vision libraries which, lacking compile-time knowledge of the particular algorithm of interest, provide parameter type generality by branching at runtime.
This branching is often unnecessary for any particular algorithm where all runtime samples are expected to follow the same code path.

\section{Portability}
Likely is written in a portable subset of modern C++\cite{stroustrup1986c++}, which means it can \fref{http://llvm.org/docs/GettingStarted.html\#hardware}{run everywhere \LLVM\ does}.
Algorithms execute natively on single CPU threads via the \LLVM\ \noun{Machine Code \JIT} compiler, multi-core CPUs via a custom \noun{OpenMP}-like backend, and GPUs and coprocessors via \noun{CUDA} or \noun{OpenCL} (in progress).

Likely algorithms can be \fref{http://liblikely.org/share/likely/hello\_world/hello\_world\_static.c}{statically compiled to native object files}, \fref{http://liblikely.org/share/likely/hello\_world/hello\_world\_jit.c}{just-in-time compiled for dynamic applications and scripts}, integrated into other languages (in progress), published as web services (in progress), and even transcompiled into single \noun{C} or \noun{JavaScript} source files (in progress).

\section{Live Coding}
Developing computer vision algorithms is a creative process, and like all creative processes it relies on immediate visceral feedback through interaction with the creative medium.
Algorithms have numerous parameters, and visualizing their effects is critical to building intuition and making improvements.
The computer vision domain and Likely's library design are well suited for the idea of \dindex{live coding} popularized by \fref{http://worrydream.com/}{Bret Victor}.
In fact, Likely's accompanying IDE called \emph{dream}, is \fref{https://www.youtube.com/watch?v=a\_hz8wFACVM}{designed from the ground up to support interactive algorithm development}.

\section{Literate Programming}
Likely is a \cref{http://www.literateprogramming.com/}{\pindex{literate programming}}{knuth1984literate} language.
In addition to serving as the reference manual for the programming language and standard library, \fref{\githuburl library/standard.tex}{\emph{this document is also the source code for the Likely standard library}}.

Likely recognizes both \fref{https://help.github.com/articles/github-flavored-markdown}{GitHub Flavored Markdown} (\GFM) and \cref{http://www.latex-project.org/}{\LaTeX \index{LaTeX}}{lamport1986document} syntax, and will automatically extract source code from appropriately marked blocks.

\part{Language Reference}
\chapter{Parsing}
This chapter covers how a source code file is translated into the compiler's \emph{abstract syntax tree} (\AST) program representation.
This is also referred to as the compiler's \fref{\awsurl doxygen/group\_\_frontend.html}{\dindex{frontend}}.

\section{Source Code File Formats}
To facilitate \pindex{literate programming}, Likely recognizes several source code file formats based on their extension:

\begin{figure}[h]
\begin{description}
\item[.lisp] - The entire file is Likely source code.
\item[.md]   - \GFM\ file, source code is extracted from \texttt{likely} syntax-highlighted code blocks.
\item[.tex]  - \pindex{LaTeX} file, source code is extracted from the \texttt{likely} environment.
\end{description}
\caption{Source code file extensions and formats.}
\end{figure}

For example, this \GFM\ \fref{\githuburl library/literate\_programming.md}{source file} can be rendered \fref{http://liblikely.org/?href=literate\_programming}{in a web browser} and also executed by Likely:

\begin{lstlisting}[language=bash, caption={Literate programming using \GFM.}, captionpos=b]
$ likely library/literate_programming.md
42
\end{lstlisting}

Similarly, this \pindex{LaTeX} \fref{\githuburl library/literate\_programming.tex}{source file} can be rendered \fref{\awsurl latex/literate\_programming.pdf}{to \noun{PDF}} and also executed by Likely:

\begin{lstlisting}[language=bash, caption={Literate programming using \pindex{LaTeX}.}, captionpos=b]
$ likely library/literate_programming.tex
42
\end{lstlisting}

\tindex{likely\char`_guess\char`_file\char`_type} is the function call for inferring file format from extension.

\section{Lexical Analysis}
Source code is first converted into a sequence of words or \dindex{tokens} by applying the following rules:

\begin{figure}[h]
\begin{enumerate}
\item Tokens are separated by \pindex{whitespace}.
\item The following special single-character tokens need not be separated by \pindex{whitespace}:\newline
      \texttt{();.:}
\item The characters between double-quotation marks are a single token, including the quotation marks themselves.
\end{enumerate}
\caption{Lexical analysis rules.}
\end{figure}

Where \dindex{whitespace} is defined by the C function \fref{http://www.cplusplus.com/reference/cctype/isspace/}{\texttt{isspace}}.

\tindex{likely\char`_lex} is the function call for lexical analysis.

\section{Prefix Notation}
Likely is a member of the Lisp family of languages, accepting fully-parenthesized Polish \pindex{prefix notation}:

\begin{figure}[h]
\centering
\texttt{($operator$ $operand_1$ $operand_2$ \dots\ $operand_N$)}
\caption{Prefix notation.}
\end{figure}

For example:

\begin{lstlisting}[language=bash, caption={Prefix notation example.}, captionpos=b]
$ likely -c "(+ 1 2)"
3
\end{lstlisting}

An important property of prefix notation is that it is equivalent to the \AST, where a branch is a \dindex{list} and leaf is an \dindex{atom}.
Every branch or leaf is also called an \dindex{expression}.
A reading of \cref{https://mitpress.mit.edu/sicp/full-text/book/book.html}{Structure and Interpretation of Computer Programs}{Abelson:1996:SIC:547755} is strongly encouraged for developers interested in a deeper understanding of the motivations behind this syntax.

\tindex{likely\char`_parse} is the function call for constructing an \AST\ from \pindex{tokens}.

\section{Special Tokens}
Likely recognizes three special \pindex{tokens} that influence how \pindex{expressions} are parsed into an \AST.
These \pindex{tokens} allow developers to selectively depart from \pindex{prefix notation} to improve code readability.

\begin{table}[h]
\centering
\begin{tabular}{@{} l l @{}}
\toprule
Token & Name    \\ \midrule
;     & Comment \\
.     & Compose \\
:     & Infix   \\
\bottomrule
\end{tabular}
\caption{Likely special tokens.}
\end{table}

When in doubt, you can print the \AST\ to see how source code is parsed:
\begin{lstlisting}[language=bash, caption={Printing the abstract syntax tree.}, captionpos=b]
$ likely -ast -c "1 :+ 2"
(+ 1 2)
\end{lstlisting}

\subsection{Comment (;)}
The semicolon and all subsequent \pindex{tokens} through the end of the line are excluded from the \AST.

\begin{verbatim}
(this is some code) ; this is a comment
; (also a comment)
(back to code again)
\end{verbatim}

\subsection{Compose (.)}
The \pindex{expression} to the left-hand-side (LHS) of the period is the first operand of the expression to the right-hand-side (RHS) of the period.
Compose \index{compose} is \emph{left-associative}.

\begin{verbatim}
x.f     ; Parsed as (f x)
x.f.g   ; Parsed as (g (f x))
x.(f y) ; Parsed as (f x y)
(f x).g ; Parsed as (g (f x))
(g x.f) ; Parsed as (g (f x))
7.2     ; Parsed as 7.2
3.sq    ; Evaluates to 9
1.(+ 2) ; Evaluates to 3
\end{verbatim}

We might call the third example \cref{http://www.drdobbs.com/cpp/uniform-function-call-syntax/232700394}{uniform function call syntax}{Bright:2012:UFC}.
In the sixth example, note how this transformation does not apply to numbers!

\subsection{Infix (:)}
\label{subsec:infix}
The \pindex{expression} to the RHS of the colon is the operator.
The \pindex{expression} to the LHS of the colon is the first operand.
The second \pindex{expression} to the RHS of the colon is the second operand.
Infix \index{infix} is \emph{right-associative} and has \emph{lower} precedence than \pindex{compose}.

\begin{verbatim}
x:f y       ; Parsed as (f x y)
z:g x:f y   ; Parsed as (g z (f x y))
x:f (g y)   ; Parsed as (f x (g y))
x.f:h y.g   ; Parsed as (h (f x) (g y))
(g x:f y)   ; Parsed as (g (f x y))
1:+ 2       ; Evaluates to 3
3.sq:+ 4.sq ; Evaluates to 25
\end{verbatim}

\section{Imperative Notation (\{ \dots\ \})}
Likely makes one additional departure from fully parenthesized notation, automatically identifying braces as a \pindex{list} structure.

\begin{verbatim}
{ x y z } ; Parsed as ({ x y z })
\end{verbatim}

See subsection \ref{subsec:begin} for the semantics of this structure.

\chapter{Literals}
The Likely compiler recognizes the following kinds of fixed, or \dindex{literal}, \pindex{atom}s.

\section{Integer}
An \dindex{integer} is an \pindex{atom} that can be parsed \emph{to completion in base-10} by the C function \fref{http://www.cplusplus.com/reference/cstdlib/strtoll/}{\texttt{strtoll}}.
If the value can be represented without loss as an \texttt{int32\_t} then it is done so, otherwise it is represented as an \texttt{int64\_t}.

\begin{verbatim}
-42 ; An integer
42a ; Not an integer
\end{verbatim}

\section{Real}
A \dindex{real} is an \pindex{atom} that can be parsed \emph{to completion} by the C function \fref{http://www.cplusplus.com/reference/cstdlib/strtod/}{\texttt{strtod}}.
If the value can be represented without loss as a \texttt{float} then it is done so, otherwise it is represented as a \texttt{double}.

\begin{verbatim}
0.42  ; A real
0.42- ; Not a real
\end{verbatim}

\section{String}
A \dindex{string} is an \pindex{atom} enclosed in double-quotation marks.
Values are represented as \texttt{const char*}.

\begin{verbatim}
"hello world" ; A string
'hello world' ; Not a string
\end{verbatim}

\section{Type}
A \dindex{type} is a keyword indicating how data is represented.
Let's start with a few examples:

\begin{verbatim}
i16  ; 16-bit signed integer scalar
u32  ; 32-bit unsigned integer scalar
f64  ; 64-bit floating-point real scalar
f32C ; 32-bit floating-point real multi-channel vector
u8XY ; 8-bit unsigned integer multi-column multi-row matrix
\end{verbatim}

In general, a \pindex{type} conforms to the following regular expression:

\begin{figure}[h]
\centering
\texttt{[uif]\char`\\ d+P?S?C?X?Y?T?}
\caption{Regular expression for identifying types.}
\end{figure}

The first character indicates the element bit pattern, and is one of:

\begin{figure}[h]
\begin{description}
\item[u] - Unsigned integer
\item[i] - Signed integer
\item[f] - Floating-point real
\end{description}
\caption{Type leading character.}
\end{figure}

The next one-or-more decimal characters indicate the element bit depth, and should generally be a power of two.
The remaining capitalized characters indicate:

\begin{figure}[h]
\begin{description}
\item[P] - Type is a pointer
\item[S] - Perform saturated arithmetic
\item[C] - Multi-channel matrix
\item[X] - Multi-column matrix
\item[Y] - Multi-row matrix
\item[T] - Multi-frame matrix
\end{description}
\caption{Type trailing characters.}
\end{figure}

The presence or absense of \texttt{C}, \texttt{X}, \texttt{Y} and \texttt{T} indicate whether we call it a \dindex{scalar}, \dindex{vector} or \dindex{matrix} \pindex{type}.

\tindex{likely\char`_type\char`_from\char`_string} is the formal definition of a \pindex{type}.
In addition to accepting the general pattern mentioned above, this function also recognizes a number of special cases corresponding to values in \tindex{likely\char`_type\char`_mask}.
The special cases include all common C types.

\subsection{First-class Citizenship}
An important aspect of a \pindex{type} is that, contrary to most programming languages, it is a \dindex{first-class citizen} of Likely.
In addition to having an integer value represented in the API by \tindex{likely\char`_type}, it can also be used as a unary-operand to perform a cast.
Most importantly though, it can be \emph{passed as an argument to a function}.

\subsection{Implicit Conversion}
Intrinsic \pindex{operator}s will implictly convert their \pindex{operand}s to the same type.
The rules for doing so are as follows:

\begin{figure}[h]
\begin{enumerate}
\item{If two operands differ in depth, the operand with lesser \pindex{depth} is promoted to the greater \pindex{depth}.}
\item{If one of the operands is an \pindex{integer} and the other is a \pindex{real}, the \pindex{integer} is casted to a \pindex{real}.}
\item{If both operands are integers and one operand is \pindex{signed}, the operation is \pindex{signed}.}
\item{If one operand is \pindex{saturated}, the operation is \pindex{saturated}.}
\end{enumerate}
\caption{Implicit conversion rules.}
\end{figure}

These rules are codified in the function \tindex{likely\char`_type\char`_from\char`_types}.

\section{File-type}
If a \pindex{type} indicates how to interpret the contents of an address in memory, a \dindex{file-type} indicates how to interpret the contents of a file on disk.
This is a relatively unimportant portion of the language, and is generally transparent to the developer.
Therefore it is just noted that, analogous to a \pindex{type}, a \pindex{file-type} is defined by the function \tindex{likely\char`_file\char`_type\char`_from\char`_string}, and has an integer value represented in the API by \tindex{likely\char`_file\char`_type}.

\section{\texttt{this}}
The \dindex{\texttt{this}} keyword refers to the appliation's current \dindex{environment}, or set of referenceable variables.
Likely's ability to self-execute code, such as when including the contents of another file, are made possible by \tindex{this}.

In terms of the Likely API, \texttt{this} is a pointer to a \tindex{likely\char`_environment}, and is most useful in conjunction with the function \tindex{likely\char`_eval}.

\chapter{Intrinsic Operators}
An \dindex{intrinsic operator} is the basic construct for higher order expressions on \pindex{literal}s.
Unless otherwise noted, the choice of symbols and their definitions are consistent with C.
Likely has the following builtin operators.

\section{Arithmetic}
\subsection{(+ \textit{a b})}
The \dindex{addition} of $a$ and $b$.

\begin{verbatim}
(+ 2 2)     ; Evaluates to 4
(+ 1.8 2)   ; Evaluates to 3.8
(+ 1.8 2.2) ; Evaluates to 4.0
\end{verbatim}

\subsection{(- \textit{lhs [rhs]})}
If $rhs$ is provided then the \dindex{subtraction} of $lhs$ by $rhs$, otherwise the \dindex{negation} of $lhs$.

\begin{verbatim}
(- 3 2)     ; Evaluates to 1
(- 2 3)     ; Evaluates to -1
(- 3.2 2)   ; Evaluates to 1.2
(- 3.2 2.2) ; Evaluates to 1.0
(- 1)       ; Evaluates to -1
(- -1.1)    ; Evaluates to 1.1
\end{verbatim}

\subsection{(* \textit{a b})}
The \dindex{multiplication} of $a$ and $b$.

\begin{verbatim}
(* 1 2)     ; Evaluates to 2
(* 1.3 2)   ; Evaluates to 2.6
(* 1.5 2.0) ; Evaluates to 3.0
\end{verbatim}

\subsection{(/ \textit{num denom})}
The \dindex{division} of $num$ by $denom$.

\begin{verbatim}
(/ 4 2)     ; Evaluates to 2
(/ 4.5 2)   ; Evaluates to 2.25
(/ 4.2 2.1) ; Evaluates to 2.0
\end{verbatim}

\subsection{(\% \textit{num denom})}
The \dindex{remainder} after division of $num$ by $denom$.

\begin{verbatim}
(% 7 3)      ; Evaluates to 1
(% 6 3)      ; Evaluates to 0
(% 6 3.0)    ; Evaluates to 0.0
(% 6.5 3)    ; Evaluates to 0.5
(% -6.5 3)   ; Evaluates to -0.5
(% 6.5 -3)   ; Evaluates to 0.5
(% -6.5 -3)  ; Evaluates to -0.5
(% 6.5 7.5)  ; Evaluates to 6.5
(% 6.5 3.25) ; Evaluates to 0.0
\end{verbatim}

\section{Comparison}
The output of these operators is always a 1-bit \pindex{integer}.

\subsection{(== \textit{a b})}
\texttt{1} if $a$ is \dindex{equal-to} $b$, \texttt{0} otherwise.

\begin{verbatim}
(== 2 2)   ; Evaluates to 1
(== 2 2.0) ; Evaluates to 1
(== 2 -2)  ; Evaluates to 0
(== 2 2.1) ; Evaluates to 0
\end{verbatim}

\subsection{(!= \textit{a b})}
\texttt{1} if $a$ is not \pindex{equal-to} to $b$, \texttt{0} otherwise.

\begin{verbatim}
(!= 3 3)   ; Evaluates to 0
(!= 3 3.0) ; Evaluates to 0
(!= 3 -3)  ; Evaluates to 1
(!= 3 3.1) ; Evaluates to 1
\end{verbatim}

\subsection{(< \textit{lhs rhs})}
\texttt{1} if $lhs$ is \dindex{less-than} $rhs$, \texttt{0} otherwise.

\begin{verbatim}
(< 4 5)    ; Evaluates to 1
(< 4 -5.0) ; Evaluates to 0
(< 4 4.0)  ; Evaluates to 0
\end{verbatim}

\subsection{(<= \textit{lhs rhs})}
\texttt{1} if $lhs$ is \pindex{less} than or \pindex{equal-to} $rhs$, \texttt{0} otherwise.

\begin{verbatim}
(<= 4 5)    ; Evaluates to 1
(<= 4 -5.0) ; Evaluates to 0
(<= 4 4.0)  ; Evaluates to 1
\end{verbatim}

\subsection{(> \textit{lhs rhs})}
\texttt{1} if $lhs$ is \dindex{greater-than} $rhs$, \texttt{0} otherwise.

\begin{verbatim}
(> 6 7)    ; Evaluates to 0
(> 6 -7.0) ; Evaluates to 1
(> 6 6.0)  ; Evaluates to 0
\end{verbatim}

\subsection{(>= \textit{lhs rhs})}
\texttt{1} if $lhs$ is \pindex{greater} than or \pindex{equal-to} $rhs$, \texttt{0} otherwise.

\begin{verbatim}
(>= 6 7)    ; Evaluates to 0
(>= 6 -7.0) ; Evaluates to 1
(>= 6 6.0)  ; Evaluates to 1
\end{verbatim}

\section{Bit Manipulation}
\subsection{(\& \textit{a b})}
The \dindex{bitwise-and} of $a$ and $b$.

\begin{verbatim}
(& 1 2) ; Evaluates to 0
(& 1 3) ; Evaluates to 1
\end{verbatim}

\subsection{(\textbar\ \textit{a b})}
The \dindex{bitwise-or} of $a$ and $b$.

\begin{verbatim}
(| 1 2) ; Evaluates to 3
(| 1 3) ; Evaluates to 3
\end{verbatim}

\subsection{(\textasciicircum\ \textit{a b})}
The \dindex{bitwise-exclusive-or} of $a$ and $b$.

\begin{verbatim}
(^ 1 2) ; Evaluates to 3
(^ 1 3) ; Evaluates to 2
\end{verbatim}

\subsection{(<< \textit{x n})}
The \dindex{left-shift} of $x$ by $n$ bits.

\begin{verbatim}
(<< 2 0) ; Evaluates to 2
(<< 2 1) ; Evaluates to 4
\end{verbatim}

\subsection{(>> \textit{x n})}
The \dindex{right-shift} of $x$ by $n$ bits.
If $x$ is signed then the operation is an \dindex{arithmetic} right-shift (sign extension), otherwise it is a \dindex{logical} right-shift (zero extension).

\begin{verbatim}
(>> 2 0)  ; Evaluates to 2
(>> 2 1)  ; Evaluates to 1
(>> 2 2)  ; Evaluates to 0
(>> -2 0) ; Evaluates to -2
(>> -2 1) ; Evaluates to -1
\end{verbatim}

\section{Math}
A subset of the \fref{http://www.cplusplus.com/reference/cmath/}{C common mathematical operations}.
The output of these operators is always a \pindex{real}.

\subsection{(sqrt \textit{x})}
The \dindex{square root} of $x$.

\begin{verbatim}
(sqrt 2.0) ; Evaluates to 1.41421
(sqrt 4)   ; Evaluates to 2.0
(sqrt 0)   ; Evaluates to 0.0
\end{verbatim}

\subsection{(sin \textit{x})}
The \dindex{sine} of an angle of $x$ radians.

\begin{verbatim}
(sin 0)         ; Evaluates to 0.0
(sin 1.570796)  ; Evaluates to 1.0
(sin -1.570796) ; Evaluates to -1.0
(sin 0.523599)  ; Evaluates to 0.5
\end{verbatim}

\subsection{(cos \textit{x})}
The cosine of an angle of $x$ radians.

\begin{verbatim}
(cos 0)        ; Evaluates to 1.0
(cos 3.141593) ; Evaluates to -1.0
(cos 1.047198) ; Evaluates to 0.5
\end{verbatim}

\subsection{(pow \textit{base exponent})}
The $base$ raised to the \dindex{power} $exponent$.

\begin{verbatim}
(pow 2 3)     ; Evaluates to 8.0
(pow 2 -3)    ; Evaluates to 0.125
(pow -2 3)    ; Evaluates to -8.0
(pow 1.5 0.5) ; Evaluates to 1.22474
(pow 2 0.5)   ; Evaluates to 1.41421
(pow 4 0.5)   ; Evaluates to 2.0
(pow 4 0)     ; Evaluates to 1.0
\end{verbatim}

\subsection{(exp \textit{x})}
The \dindex{base-e exponential} function of $x$, which is $e$ raised to the \pindex{power} $x$.

\begin{verbatim}
(exp 0)   ; Evaluates to 1.0
(exp 1)   ; Evaluates to 2.71828
(exp 1.5) ; Evaluates to 4.48169
\end{verbatim}

\subsection{(exp2 \textit{x})}
The \dindex{base-2 exponential} function of $x$, which is $2$ raised to the \pindex{power} $x$.

\begin{verbatim}
(exp2 0)   ; Evaluates to 1.0
(exp2 1)   ; Evaluates to 2.0
(exp2 0.5) ; Evaluates to 1.41421
(exp2 3)   ; Evaluates to 8.0
\end{verbatim}

\subsection{(log \textit{x})}
The \dindex{natural logarithm} (base-$e$) of $x$.

\begin{verbatim}
(log 1)        ; Evaluates to 0.0
(log 2.718281) ; Evaluates to 1.0
(log 7.389056) ; Evaluates to 2.0
(log 0.5)      ; Evaluates to -0.693147
\end{verbatim}

\subsection{(log10 \textit{x})}
The \dindex{common logarithm} (base-$10$) of $x$.

\begin{verbatim}
(log10 1)   ; Evaluates to 0.0
(log10 10)  ; Evaluates to 1.0
(log10 100) ; Evaluates to 2.0
(log10 0.5) ; Evaluates to -0.30103
\end{verbatim}

\subsection{(log2 \textit{x})}
The \dindex{binary logarithm} (base-$2$) of $x$.

\begin{verbatim}
(log2 1)   ; Evaluates to 0.0
(log2 2)   ; Evaluates to 1.0
(log2 4)   ; Evaluates to 2.0
(log2 0.5) ; Evaluates to -1.0
(log2 10)  ; Evaluates to 3.32193
\end{verbatim}

\subsection{(copysign \textit{x} \textit{y})}
The result of a \dindex{copysign} operation is a value with the magnitude of $x$ and the sign of $y$.

\begin{verbatim}
(copysign 3 -1.1) ; Evaluates to -3.0
(copysign -4.3 2) ; Evaluates to 4.3
\end{verbatim}

\subsection{(floor \textit{x})}
The result of a \dindex{floor} operation is the largest integral value that is not greater than $x$.

\begin{verbatim}
(floor 2.3)  ; Evaluates to 2.0
(floor 3.8)  ; Evaluates to 3.0
(floor 5.5)  ; Evaluates to 5.0
(floor -2.3) ; Evaluates to -3.0
(floor -3.8) ; Evaluates to -4.0
(floor -5.5) ; Evaluates to -6.0
\end{verbatim}

\subsection{(ceil \textit{x})}
The result of a \dindex{ceil} operation is the smallest integral value that is not less than $x$.

\begin{verbatim}
(ceil 2.3)  ; Evaluates to 3.0
(ceil 3.8)  ; Evaluates to 4.0
(ceil 5.5)  ; Evaluates to 6.0
(ceil -2.3) ; Evaluates to -2.0
(ceil -3.8) ; Evaluates to -3.0
(ceil -5.5) ; Evaluates to -5.0
\end{verbatim}

\subsection{(trunc \textit{x})}
The result of a \dindex{trunc} operation is the nearest integral value that is not larger in magnitude than $x$.

\begin{verbatim}
(trunc 2.3)  ; Evaluates to 2.0
(trunc 3.8)  ; Evaluates to 3.0
(trunc 5.5)  ; Evaluates to 5.0
(trunc -2.3) ; Evaluates to -2.0
(trunc -3.8) ; Evaluates to -3.0
(trunc -5.5) ; Evaluates to -5.0
\end{verbatim}

\subsection{(round \textit{x})}
The result of a \dindex{round} operation is the integral value that is nearest to $x$, with halfway cases rounded away from zero.

\begin{verbatim}
(round 2.3)  ; Evaluates to 2.0
(round 3.8)  ; Evaluates to 4.0
(round 5.5)  ; Evaluates to 6.0
(round -2.3) ; Evaluates to -2.0
(round -3.8) ; Evaluates to -4.0
(round -5.5) ; Evaluates to -6.0
\end{verbatim}

\section{Statements}
A \dindex{statement} is a value-less expression.

\subsection{(= \textit{name expression})}
The \dindex{define} statement associates a $name$ with an $expression$.
A \pindex{define} is \dindex{lazy}, $expression$ will only be evaluated if and when $name$ is referenced.
Definitions are \dindex{immutable}, though they may \dindex{shadow}.

\begin{verbatim}
x := 1
y := (+ x 1)
x := 4
y      ; Evaluates to 2
(+ x 1); Evaluates to 5
\end{verbatim}

Recall that the colon is the \pindex{infix} token for syntactic sugar described in subsection \ref{subsec:infix}.

\subsection{(<\textasciitilde\ \textit{name expression})}
The \dindex{assign} statement stack-allocates a variable with type and initial value of $expression$.
The variable can be referenced using $name$.
Assignments are \dindex{mutable} using the \pindex{reassign} statement.

Because assignment is explicitly tied to stack allocation, assignments are not allowed in the global scope.

\begin{verbatim}
{
  x :<~ 1
  x
} ; Evaluates to 1
\end{verbatim}

\subsection{(<- \textit{name expression})}
The \dindex{reassign} statement stores the value of $expression$ into the variable $name$ previously defined by an \pindex{assign} statement.

\begin{verbatim}
{
  x :<~ 1
  x :<- (+ x 1)
  x
} ; Evaluates to 2
\end{verbatim}

\section{Control Flow}
\subsection{(\{ \textit{statement statement \dots\ statement expression} \})}
\label{subsec:begin}
The \dindex{begin} operator allows imperative-style programming.
Likely will execute a series of zero-or-more \pindex{statement}s, then value of the \pindex{begin} operator is the value of $expression$ evaluated in the context of the statements.

As would be expected in an imperative programming language, the open and close braces introduce \dindex{scope} for variable declarations.

\begin{verbatim}
x := 1
y := 2
{
  x := 4
  y := 5
  (+ x y)
} ; Evaluates to 9
(+ x y) ; Evaluates to 3
\end{verbatim}

\subsection{(? \textit{condition then [else]})}
The behavior of the \dindex{if} operator depends its use:

\paragraph{(? $condition$ $then$ $else$)}
In the case that the $else$ operand is provided, the \pindex{if} operator is an \pindex{expression} with value $then$ if $condition$ is not equal to \texttt{0} and $else$ otherwise.
Implicit type conversion ensures that $then$ and $else$ are the same type.

\begin{verbatim}
x := 1
y := 2
(? (< x y) x y) ; Evaluates to 1
\end{verbatim}

\paragraph{(? $condition$ $then$)}
In the case that the $else$ operand is not provided, the \pindex{if} operator is a \pindex{statement} with no value.
The $then$ \pindex{statement} is executed if and only if $condition$ is not equal to \texttt{0}.

\begin{verbatim}
{
  x :<~ 0
  (? (== x 0) (<- x 1))
  x
} ; Evaluates to 1
\end{verbatim}

\paragraph{(? \texttt{constant} $then$ $[else]$)}
In the case that the condition is a constant \pindex{integer} or \pindex{real}, the branching will happen at compile time.

\begin{verbatim}
x := 0
(? x 1 2) ; Evaluates to 2 at compile time
\end{verbatim}

\subsection{(\#)}
The \dindex{label} operator creates a point in the code that can be jumped to.
This primitive is necessary for constructing loops.

\begin{verbatim}
{
  end := 10
  i :<~ 1
  j :<~ 0
  loop := #
  (< i end) :?
  {
    j :<- (+ j i)
    i :<- (+ i 1)
    loop
  }
  j
} ; Evaluates to 45
\end{verbatim}

\section{Matricies}
The fundamental data structure in Likely is a four-dimensional \emph{matrix}.
In decreasing memory spatial locality order, its dimensions are: \emph{channels}, \emph{columns}, \emph{rows} and \emph{frames}.
These dimensions are often abbreviated \emph{c}, \emph{x}, \emph{y} and \emph{t}, respectively.
Ownership of matricies is managed automatically using reference counting.
Matricies are represented with the C API struct \tindex{likely\char`_matrix}.

\subsection{(channels \textit{[matrix]})}
The \dindex{channels} operator returns the number of \pindex{channels} in $matrix$.
If $matrix$ is not specified, this operator returns a function which when given a matrix returns the number of \pindex{channels} in the matrix.

\begin{verbatim}
(channels (read-image "data/misc/lenna.tiff")) ; Evaluates to 3
\end{verbatim}

\subsection{(columns \textit{[matrix]})}
The \dindex{columns} operator returns the number of \pindex{columns} in $matrix$.
If $matrix$ is not specified, this operator returns a function which when given a matrix returns the number of \pindex{columns} in the matrix.

\begin{verbatim}
(columns (read-image "data/misc/lenna.tiff")) ; Evaluates to 512
\end{verbatim}

\subsection{(rows \textit{[matrix]})}
The \dindex{rows} operator returns the number of \pindex{rows} in $matrix$.
If $matrix$ is not specified, this operator returns a function which when given a matrix returns the number of \pindex{rows} in the matrix.

\begin{verbatim}
(rows (read-image "data/misc/lenna.tiff")) ; Evaluates to 512
\end{verbatim}

\subsection{(frames \textit{[matrix]})}
The \dindex{frames} operator returns the number of \pindex{frames} in $matrix$.
If $matrix$ is not specified, this operator returns a function which when given a matrix returns the number of \pindex{frames} in the matrix.

\begin{verbatim}
(frames (read-image "data/misc/lenna.tiff")) ; Evaluates to 1
\end{verbatim}

\subsection{(data \textit{[matrix]})}
The \dindex{data} operator returns a pointer to the $matrix$ \pindex{data}.
If $matrix$ is not specified, this operator returns a function which when given a matrix returns a pointer to the matrix \pindex{data}.

This operator is useful for interoperability with external functions that operate on raw data buffers.
Developers generally shouldn't need to use it otherwise in Likely code.

\section{Abstraction}
\subsection{(try \textit{primary fallback})}
The \dindex{try} operator returns the $primary$ expression if it could be evaluated sucessfully, and the $fallback$ expression otherwise.

\begin{verbatim}
(eval (+ 1 1) 3) ; Evaluates to 2
(eval (+ 1) 3)   ; Evaluates to 3
\end{verbatim}

\subsection{(extern \textit{return-type symbol-name parameters})}
The \dindex{extern} operator returns a callable externally defined function.

\begin{verbatim}
((extern i32 "abs" i32) -4) ; Evaluates to 4
\end{verbatim}

\part{Standard Library}
\chapter{Basic Symbols}
\section{Constants}
\begin{likely}
null  := 0
true  := 1
false := 0
e  := (f32 2.718281)
pi := (f32 3.141592)
\end{likely}

\section{Numeric Limits}
\begin{likely}
numeric-limit-max := (-> t (numeric-limit t 1))
numeric-limit-min := (-> t (numeric-limit t 0))
\end{likely}

\section{Unary Functions}
\begin{likely}
not  := (-> a (== a false))
bool := (-> a (!= a false))
sq  := (-> a (* a a))
abs := (-> a (? (< a 0) (* -1 a) a))
\end{likely}

\section{Binary Functions}
\begin{likely}
and := (-> (a b) (& a.bool b.bool))
or  := (-> (a b) (| a.bool b.bool))
xor := (-> (a b) (^ a.bool b.bool))
min := (-> (a b) (? (< a b) a b))
max := (-> (a b) (? (> a b) a b))
\end{likely}

\chapter{Library Symbols}
\section{Types}
\begin{likely}
string-t       := i8P
void-pointer-t := i8P
type-t         := u32
file-type-t    := u32
\end{likely}

\section{Matrix Information}
\begin{likely}
elements := (-> mat mat.channels :* mat.columns :* mat.rows :* mat.frames)
bytes    := (-> mat (/ (+ (* (& mat.type depth) mat.elements) 7) 8))
\end{likely}

\section{Matrix Creation}
\begin{likely}
new := (-> (return-type channels columns rows frames data)
           ((extern return-type "likely_new" (type-t u32 u32 u32 u32 void-pointer-t)) return-type channels columns rows frames data))
imitate-size := (-> (mat type) (new type mat.channels mat.columns mat.rows mat.frames null))
imitate := (-> mat (imitate-size mat mat.type))
\end{likely}

\section{Matrix I/O}
\begin{likely}
read   := (-> (file-name file-type return-type)
              ((extern return-type "likely_read" (string-t file-type-t type-t)) file-name file-type return-type))
write  := (extern u8CXYT "likely_write" (u8CXYT string-t))
decode := (-> (mat return-type)
              ((extern return-type "likely_decode" (u8CXYT type-t)) mat return-type))
encode := (extern u8CXYT "likely_encode" (u8CXYT string-t))
render := (extern u8CXYT "likely_render" (u8CXYT f64P f64P))
show   := (extern u8CXYT "likely_show" (u8CXYT string-t))
guess-file-type := (extern file-type-t "likely_guess_file_type" string-t)
\end{likely}

\begin{likely}
read-image               := (-> file-name file-name.(read media     image          ))
read-image-grayscale     := (-> file-name file-name.(read media     image-grayscale))
read-video               := (-> file-name file-name.(read media     video          ))
read-video-grayscale     := (-> file-name file-name.(read media     video-grayscale))
read-directory           := (-> file-name file-name.(read directory video          ))
read-directory-grayscale := (-> file-name file-name.(read directory video-grayscale))
\end{likely}

\section{Compiler Frontend}
\begin{likely}
lex := (extern ast "likely_lex" (string-t file-type-t))
parse := (extern ast "likely_parse" ast)
lex-and-parse := (extern ast "likely_lex_and_parse" (string-t file-type-t))
\end{likely}

\section{Compiler Backend}
\begin{likely}
eval := (extern env "likely_eval" (ast env void-pointer-t void-pointer-t))
\end{likely}

\section{Import}
\begin{likely}
import-string := (-> (source-code source-type environment) (lex-and-parse source-code source-type).(eval environment null null))
import := (-> (file-name environment) file-name.(read guess text).data.(import-string file-name.guess-file-type environment))
\end{likely}

\section{Type Conversion}
\begin{likely}
cast := (-> (a b) (b.type a)) ; convert a to the type of b
depth-double := (-> t (| (* (& t depth) 2) (& t (~ depth))).make-type)
depth-atleast := (-> (t bits) (| (max bits (& t depth)) (& t (~ depth))).make-type)
depth-atleast-16 := (-> t (depth-atleast t 16))
depth-atleast-32 := (-> t (depth-atleast t 32))
depth-atleast-64 := (-> t (depth-atleast t 64))

single-dimension := (-> d (-> t (& t (~ d))))
single-channel := (single-dimension multi-channel)
single-column  := (single-dimension multi-column )
single-row     := (single-dimension multi-row    )
single-frame   := (single-dimension multi-frame  )

imitate-dimension := (-> d (-> (t u) (? (& u d) (| t d) (& t (~ d)))))
imitate-channel := (imitate-dimension multi-channel)
imitate-column  := (imitate-dimension multi-column)
imitate-row     := (imitate-dimension multi-row)
imitate-frame   := (imitate-dimension multi-frame)
\end{likely}

\chapter{Pixel-wise operators}
\section{Thresholding}
\begin{likely}
threshold-binary          := (-> (input threshold response) (? (> input threshold) response  0))
threshold-binary-inverse  := (-> (input threshold response) (? (> input threshold) 0         response))
threshold-truncate        := (-> (input threshold)          (? (> input threshold) threshold input))
threshold-to-zero         := (-> (input threshold)          (? (> input threshold) input     0))
threshold-to-zero-inverse := (-> (input threshold)          (? (> input threshold) 0         input))
\end{likely}

\chapter{Loops}
\section{iter}
\begin{likely}
iter :=
  (expr end) :->
  {
    i :<~ 0
    loop := #
    (!= i end) :? {
      (expr i)
      i :<- (+ i 1)
      loop
    }
  }
\end{likely}

\chapter{Reductions}
\section{Average}
\begin{likely}
average-frame :=
  mat :->
  {
    result := (new mat.type.single-frame mat.channels mat.columns mat.rows 1 null)
    len := mat.frames
    (result mat len) :=>
    {
      j :<~ (mat.type.depth-double.depth-atleast-32 0)
      (-> t (<- j (+ j (mat c x y t)))).(iter len)
      result :<- (result.type (/ j len))
    }
  }

center-frame :=
  mat :->
  {
    avg := mat.average-frame
    result := mat.(imitate-size mat.type.depth-double.signed)
    (result mat avg) :=> (<- result (- (result.type mat) (result.type avg)))
  }
\end{likely}

\chapter{Convolutions}
\section{Covariance}
\begin{likely}
covariance :=
  vecs :->
    {
        cov := (new vecs.type.floating vecs.channels vecs.columns vecs.columns vecs.frames null)
        ; TODO - complete
        cov
    }
\end{likely}

\bibliographystyle{plain}
\bibliography{library/standard}

\printindex

\end{document}
